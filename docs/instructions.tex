\documentclass{article}
\usepackage{times}
\usepackage{hyperref}
\usepackage{amsmath}
\usepackage{todonotes}

\begin{document}

  \title{Hello GitHub Classroom Assignment \\ Instructions}
  \author{bucs220@github}
  \date{\today}
  \maketitle

\section{Instructions}
\begin{itemize}
  \item Answer the questions individually. Group effort is not allowed. If you utilize code you found online or in print, you must cite its origin within your source code in accordance with the class syllabus' academic honesty and integrity policies.
  \item Solutions must be committed to your assignment repository on Github.
  \item Ensure that your code runs on \texttt{remote.cs.binghamton.edu}.
  \item Useful resources:
  \begin{itemize}
    \item \href{http://www.informit.com/blogs/blog.aspx?uk=The-10-Most-Important-Linux-Commands}{Common Linux Commands}
    \item \url{http://c-faq.com/}
    \item \url{https://cdecl.org}
  \end{itemize}
\end{itemize}

\section{Questions}
\subsection{Fibonacci Elements}
\subsubsection{Description}
The Fibonacci Sequence is the sequence of numbers defined by the following linear recurrence relation:

\begin{equation}
  \left \{     F_n     \right \}_{n=1}^{\infty } = \begin{cases}
    0 & F_1 \\
    1 & F_2 \\
    F_{n-1} + F_{n-2} & F_n
  \end{cases}
\end{equation}

Write the function \texttt{int isFib(unsigned long i)} such that given an integer $i$, the function returns $n$, where $i$ is $F_n$ if it is a Fibonacci number. 
If $i$ is not a Fibonacci number, then the function returns -1.
 Assume that {\it i} is in the range $ 0 \le i \le 1000000000$.

\subsubsection{Notes}
\begin{itemize}
  \item Note that our sequence is 1-indexed and begins with 0 as the first element. This deviates from definitions of the sequence that you may find elsewhere.
  \item Note that both $F_2$ and $F_3$ are 1. If $i$ is 1, return 2.
\end{itemize}

\subsubsection{Examples}

\begin{center}
  \label{fibexamples}
  \begin{tabular}{|l|l|}
    \hline
    {\it i} & {\it n} \\
    \hline
    0 & 1 \\
    1 & 2 \\
    2 & 4 \\
    4 & -1 \\
    \hline
  \end{tabular}
\end{center}

\end{document}
